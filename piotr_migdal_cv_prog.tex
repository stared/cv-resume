
%________________________________________________________________________________________
% @brief    LaTeX2e Resume for Piotr Migdał
% @author   Piotr Migdał
% @url      http://migdal.wikidot.com/en
% @date     November 2009
% @info     Based on Latex Resume Template by Chris Paciorek 
%           http://www.biostat.harvard.edu/~paciorek/

%________________________________________________________________________________________
\documentclass[margin,line]{resume}
\topmargin -5mm

\usepackage[utf8]{inputenc}
\usepackage[plmath,OT4]{polski}

\RequirePackage{color,graphicx}
\usepackage[usenames,dvipsnames]{xcolor}
\usepackage{hyperref}
\definecolor{linkcolour}{rgb}{0,0.2,0.6}
%\definecolor{linkcolour}{rgb}{0,0,0}
\hypersetup{colorlinks,breaklinks,urlcolor=linkcolour, linkcolor=linkcolour}
%\usepackage{fontspec} 
%\defaultfontfeatures{Scale=MatchLowercase,Mapping=tex-text} % pozwala na -- i ,,''
%\setromanfont{Gentium}

\newcommand{\superscript}[1]{\ensuremath{^{\textrm{#1}}}}
\newcommand{\thh}[0]{\superscript{th}}
\newcommand{\st}[0]{\superscript{st}}
\newcommand{\nd}[0]{\superscript{nd}}
\newcommand{\rd}[0]{\superscript{rd}}

%\sectionwidth{10}
% resumewidth	

\begin{document}
%\title{Curriculum Vitae}
%\name{\Large Curriculum Vitae}
\name{\Large Resume}

\begin{resume}

    %____________________________________________________________________________________
    % Personal Information
    \section{\mysidestyle Personal\\Information}\vspace{2mm}

    \begin{tabular}{@{} l @{\hspace{28mm}} l}
    First name / Surname:    & Piotr Migdał             \\
    Date of birth:           & 1986-03-13               \\
    Citizenship:             & Polish                   \\ 
    E-mail:                  & \href{pmigdal@gmail.com}{\tt pmigdal@gmail.com}        \\
    Homepage:			& \href{http://migdal.wikidot.com/en}{\tt http://migdal.wikidot.com/en} \\
    Phone:                   & +34 644 226 536
    \end{tabular}
\vspace{3mm}

   % ____________________________________________________________________________________
   %  Objective
    \section{\mysidestyle Objective}
    A PhD student in theoretical quantum physics looking for challenges in data science and programming. 

\vspace{3mm}

    %____________________________________________________________________________________
    % Research Interests
    %\section{\mysidestyle Research\\Interests}
    %quantum optics, quantum information, complex systems, mathematical modelling in psychology

%\vspace{3mm}

    %____________________________________________________________________________________
    % Education
    \section{\mysidestyle Education}
    
    {\bf ICFO - The Institute of Photonic Sciences},  Castelldefels (Barcelona), Spain \\%
    Quantum Optics Theory Group, advisor: prof. Maciej Lewenstein \hfill {\bf Feb 2011 -- }\\
   \begin{list2}
        \vspace*{-4mm}
        \item Symmetries and self-similarities of quantum states, application of sequence-analysis methods for quantum states.
    \end{list2}

    {\bf University of Warsaw}, Warsaw, Poland \\
    {\sl Inter-faculty Studies in Mathematics and Natural Sciences} \hfill {\bf 2005 -- 2011}\\
    \begin{list2}
      \vspace*{-4mm}
      \item Master's Degree  (5 year programme) {\hfill Jan 2011}\\
      Physics (Theoretical Physics --- Quantum Optics and Atomic Physics),
      \href{http://migdal.wikidot.com/en:collective-decoherence}{thesis}, grade: 5/5
      \item Bachelor's Degree, Mathematics (Game Theory),
      \href{http://migdal.wikidot.com/en:mafia}{thesis}, grade: 5/5 \hfill Sept 2009
    \end{list2}
    
\vspace{3mm}

    %____________________________________________________________________________________
    % Research Experience
    \section{\mysidestyle Research\\Experience}

	{\bf  Interdisciplinary Centre for Mathematical and Computational Modelling, University of Warsaw}, Warsaw, Poland\\
	dr Dariusz Plewczyński Group, mathematical psychology \hfill {\bf Nov 2010 -- }\\
    \begin{list2}
        \vspace*{-4mm}
        \item Cognitive computing: human information sharing models, models of human perception.
    \end{list2}

	{\bf Institute of Theoretical Physics, University of Warsaw}, Warsaw, Poland\\
	prof. Konrad Banaszek Group, quantum optics (theoretical physics) ,
	TEAM Programme operated by the Foundation for Polish Science \hfill {\bf Jan -- Sep 2010}\\
    \begin{list2}
        \vspace*{-4mm}
        \item Quantum-enhanced protocols in realistic environments: Generation schemes for robust entangled states.
    \end{list2}

    {\bf ICFO - The Institute of Photonic Science},\\ Castelldefels (Barcelona), Spain\\
    prof. Maciej Lewenstein Group, quantum optics (theoretical physics) \hfill {\bf Oct 2009}\\
    \begin{list2}
        \vspace*{-4mm}
        \item Ultra-cold atomic gases in non-abelian gauge fields.
    \end{list2}

    {\bf Faculty of Physics, Nicolaus Copernicus University}, Toruń, Poland\\
    prof. Konrad Banaszek Group, quantum optics (theoretical physics) \hfill {\bf 2007 -- 2009}\\
    \begin{list2}
        \vspace*{-4mm}
        \item Analysis of spontaneous parametric down-conversion, estimation of the quantum noise.
        \item Averaging procedures for generating decoherence-free states.
    \end{list2}

\newpage

    {\bf Faculty of Physics, University of Warsaw}, Warsaw, Poland\\
    prof. Czesław Radzewicz Group, applied optics (experimental physics) \hfill {\bf 2006 -- 2007}\\
    \begin{list2}
        \vspace*{-4mm}
        \item Wavefront sensor with Fresnel zone plates for use in an undergraduate laboratory.
    \end{list2}

    {\bf Francis Bitter Magnet Laboratory, Massachusetts Institute of Technology},\\
    Jagadeesh Moodera Group, spintronics (experimental physics) \hfill Cambridge (MA), USA\\
    during the Research Science Institute 2005 scholarship \hfill {\bf Jul –- Aug 2005}\\
    \begin{list2}
        \vspace*{-4mm}
        \item Experimental work on organic tunnelling barriers.
    \end{list2}

\vspace{3mm}


    %____________________________________________________________________________________
    % Activities
    \section{\mysidestyle Projects,\\Activities}
    Teaching, popularization and organizational work
        \begin{list2}
	    \item a moderator of \href{http://theoreticalphysics.stackexchange.com/}{Theoretical Physics - Stack Exchange}\hfill Nov 2011--May 2012
        \item prepared and lead a month-length course {\sl Introduction to quantum cryptography} for 2 talented high-school students for  Jovenes y Ciencia programme by Caixa Catalunya\hfill Jun 2011
        \item co-organizer of the \href{http://warsztatywww.wikidot.com/en}{6th and 7th  Summer Scientific Schools} \hfill 2010-2011%\\
%        \item co-organizer of \href{http://www.flaszki.waw.pl/}{Flaszki} (a series of 5-min talks) \hfill 2010
%        (non-profit camp for gifted high school students)
%        \item inter alia co-founder and president of the \href{http://skfiz.fuw.edu.pl/en}{Physics Students' Society}, Univ. of Warsaw \hfill 2006--
%        \item member of the {\sl Neurobiology Students' Scientific Society}, Univ. of Warsaw  \hfill 2008--
        \item voluntary tutor of the \href{http://www.fundusz.org/?lang=gb}{Polish Children's Fund} during 9 scientific workshops \hfill 2006--2011\\
        (for gifted high school individuals), the last one: \href{http://migdal.wikidot.com/fizyka-stada}{Physics of Herd}
        \item 9 talks on students' conferences (physics, mathematics, psychology) \hfill 2006--2010
        \item 5 scientific and didactic shows \hfill 2006--2010
		\item mentor of A. Kubica and W. Pilewski, winners (1st prize) of the \href{http://www.eucys09.fr/}{21st European Union Contest for Young Scientists in Paris}  \hfill 2009
%            in the {\sl Polish Finals of the European Union Contest for Young Scientists} for its participants (Aleksander Kubica, Wiktor Pilewski) and in 1st in the 21st {\sl EUCYS} in Paris
        \item head organizer of the 7th Polish Physics Students' Societies Conference \hfill 2007--2009\\
        (7--10.11.2008, University of Warsaw, 80 participants from 15 universities, 36 talks)
         \item problem setter of the Polish Physics Olympiad \hfill 2006--2008
        \end{list2} 
	
	Courses and conferences --- participant 
	        \begin{list2}
	    \item \href{http://www.eccs2011.eu/}{ECCS2011}: European Conference of Complex Systems and two satellite events: \href{http://markov.uc3m.es/complexdynamics11/Home.html}{Complex Dynamics of Human Interactions} and \href{http://cssociety.org/PhDVienna2011}{PhD 'Research in Progress' Workshop}  \hfill 2011
	    \item \href{http://netsci2011.net/}{NetSci 2011}: The International School and Conference on Network Science\hfill 2011
	    \item \href{http://bss.mafihe.hu/}{Balaton Summer School in Physics}: Self-organization and complex systems\hfill 2010
        \item {Summer Course: Quantum Engineering, Advanced Level} \hfill 2009
 %       (31.08-25.09.2009, Faculty of Physics, University of Warsaw)
        \item {Quantum Optics and Quantum Information}\hfill 2008 
 %       (22-25.08.2008, Faculty of Physics, Nicolaus Copernicus University, Toruń)  
        \item 7 scientific workshops organized by the \href{http://www.fundusz.org/?lang=gb}{Polish Children's Fund} \hfill 2003--2005
                \end{list2} 
       
     Other
	        \begin{list2}
	    \item co-founder of \href{http://confrenzy.com}{Confrenzy} --- a website listing scientific events \hfill Mar 2011--
                \end{list2} 

%jeszcze wWw!

    %courses
    %students conferences
    %conferences
    %didactics?

\vspace{3mm}

   %____________________________________________________________________________________
    % Awards and Scholarships
    \section{\mysidestyle Awards,\\Scholarships}
    \begin{list2}
    	\item ICFO Innovation Fund grant awarded for Confrenzy \hfill Oct 2011
        \item Scholarship of the Minister of Science and Higher Education\\for exceptional achievements in science \hfill 2007--2010
        \item 2nd  place in the Didactic Show Competition (Faculty of Physics, University of Warsaw)  \hfill 2008  %6.12.2008 
        \item Research Science Institute (Massachusetts Institute of Technology, Cambridge, MA, USA) \hfill 2005
        \item Scholarship of the Polish Children's Fund \hfill 2003--2005
        \item Scholarship of the Minister of Education, Science and Sport \hfill 2003--2005
        \item Bronze Medal in the International Physics Olympiad (Pohang, South Korea) \hfill 2004
        \item 2nd  place in the 53rd Polish Physics Olympiad \hfill 2004
        \item 7th  place in the 52nd Polish Physics Olympiad \hfill 2003
    \end{list2}

%\vspace{3mm}
\newpage

    %____________________________________________________________________________________
    % Awards and Scholarships
    \section{\mysidestyle Publications}
    Papers
    \begin{list2}
        \item 6 peer-reviewed publications (including PRA, PRL, New Jour of Phys)
        \item x essays, students' conferences and expository articles
        \item y conference talks and seminars, z posters 
    \end{list2}

\vspace{3mm}

    %____________________________________________________________________________________
    % Skills
    \section{\mysidestyle Skills}
    \begin{list2}
        \item Languages: Polish (native), English (fluent)
        \item Science-related computer languages: Mathematica, MATLAB, LabView, LaTeX, Python
        \item Systems: Linux (user), Windows (user), Mac OS X (user)
    \end{list2}

%\vspace{3mm}
    %____________________________________________________________________________________
    % Hobbies
    %\section{\mysidestyle Hobbies}
    %photography, hiking, cognitive science

%________________________________________________________________________________________
\end{resume}
\end{document}

%________________________________________________________________________________________
% EOF
