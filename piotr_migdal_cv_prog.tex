
%________________________________________________________________________________________
% @brief    LaTeX2e Resume for Piotr Migdał
% @author   Piotr Migdał
% @url      http://migdal.wikidot.com/en
% @date     November 2009
% @info     Based on Latex Resume Template by Chris Paciorek 
%           http://www.biostat.harvard.edu/~paciorek/

%________________________________________________________________________________________
\documentclass[margin,line]{resume}
\topmargin -5mm

\usepackage[utf8]{inputenc}
\usepackage[plmath,OT4]{polski}

\RequirePackage{color,graphicx}
\usepackage[usenames,dvipsnames]{xcolor}
\usepackage{hyperref}
\definecolor{linkcolour}{rgb}{0,0.2,0.6}
%\definecolor{linkcolour}{rgb}{0,0,0}
\hypersetup{colorlinks,breaklinks,urlcolor=linkcolour, linkcolor=linkcolour}
%\usepackage{fontspec} 
%\defaultfontfeatures{Scale=MatchLowercase,Mapping=tex-text} % pozwala na -- i ,,''
%\setromanfont{Gentium}

\newcommand{\superscript}[1]{\ensuremath{^{\textrm{#1}}}}
\newcommand{\thh}[0]{\superscript{th}}
\newcommand{\st}[0]{\superscript{st}}
\newcommand{\nd}[0]{\superscript{nd}}
\newcommand{\rd}[0]{\superscript{rd}}

%\sectionwidth{10}
% resumewidth	

\begin{document}
%\title{Curriculum Vitae}
%\name{\Large Curriculum Vitae}
\name{\Large Resume}

\begin{resume}

    %____________________________________________________________________________________
    % Personal Information
    \section{\mysidestyle Personal\\Information}\vspace{2mm}

    \begin{tabular}{@{} l @{\hspace{28mm}} l}
    First name / Surname:    & Piotr Migdał             \\
    E-mail:                  & \href{pmigdal@gmail.com}{\tt pmigdal@gmail.com}        \\
    Phone:                   & +34 644 226 536 (Spanish), +48 537 459 068 (Polish)\\
    Homepage:           & \href{http://migdal.wikidot.com/en}{\tt http://migdal.wikidot.com/en} \\
    Profiles: & \href{https://github.com/stared}{GitHub}, \href{http://stackexchange.com/users/506817/piotr-migdal?tab=accounts}{StackExchange}\\
    \end{tabular}
%\vspace{-2mm}

   % ____________________________________________________________________________________
   %  Objective
    \section{\mysidestyle Objective}
    A PhD student in quantum physics looking for challenges in data science and programming. 

%\vspace{-2mm}

    %____________________________________________________________________________________
    % Research Interests
    %\section{\mysidestyle Research\\Interests}
    %quantum optics, quantum information, complex systems, mathematical modelling in psychology

%\vspace{3mm}

    %____________________________________________________________________________________
    % Research Experience
    \section{\mysidestyle Experience}

    \begin{list2}

    \item Data Science Intern at {\bf \href{http://compass.co}{Startup Compass Inc.}}, San Francisco, CA, USA \hfill Jul -- Oct 2013\\
        Designing distance function for software companies, based on a multidimensional dataset.\\
        Data processing (Python: SciPy, Pandas; MongoDB), data exploration and visualization (Gephi, D3.js).

    \item Participated in {\bf \href{http://bigdive.eu}{BigDive}}, a 4-week long hands-on workshop in data science and big data processing, for 20 participants: \hfill { Oct 2012}\\
        data visualization (Gephi, D3.js), Python and JavaScript optimization, no-SQL databases (MongoDB), Hadoop (Amazon Elastic MapReduce), machine learning, NLP, working on real Twitter data; \href{https://speakerdeck.com/pmigdal/a-map-of-256-tags-of-stackoverflow-at-bigdive-wrap-up}{my wrap-up presentation} 

    \end{list2}

    \section{\mysidestyle Research}

    \begin{list2}

    \item {\bf Related expertize}: quantum mechanics, statistical mechanics, graph theory, complex sytstems, complex networks, information theory, entropy, collective decision-making, mathematical psychology

    \item {\bf Internships at}: ISI Foundation (Italy), ICFO (Spain), Univ. of Toruń (Poland), MIT (USA)
    
    \item {\bf Research papers}: \href{http://scholar.google.com/citations?user=JUwBsPAAAAAJ&hl=en}{8 publications}, \href{http://arxiv.org/a/migdal\_p\_1}{6 available openly on arXiv};
    a \href{http://www.youtube.com/watch?v=8fPAzOziTZo}{video abstract for Qubism}

    \item {\bf Conferences and workshops}: in complex systems (e.g. \href{http://www.eccs2012.eu/}{ECCS2012}, \href{http://netsci2011.net/}{NetSci2011}), quantum optics and information (e.g. \href{http://www.uibk.ac.at/th-physik/qism2012/}{QISM2012}, \href{http://qcmc2012.org/}{QCMC2012})

    
    \end{list2}

%\vspace{-2mm}

    %____________________________________________________________________________________
    % Education
    \section{\mysidestyle Education}
    
    {\bf ICFO - The Institute of Photonic Sciences},  Castelldefels (Barcelona), Spain \\%
    Quantum Optics Theory Group, advisor: prof. Maciej Lewenstein \hfill { Feb 2011-- }\\
%   \begin{list2}
%        \vspace*{-4mm}
%        \item Symmetries and self-similarities of quantum states, application of sequence-analysis methods for quantum states.
%    \end{list2}
    {\bf University of Warsaw}, Warsaw, Poland (MISMaP program) \hfill { 2005--2011}\\
    % {\sl Inter-faculty Studies in Mathematics and Natural Sciences} \\
    \begin{list2}
      \vspace*{-4mm}
      \item Master's Degree  (5 year program),
      Theoretical Physics,
      \href{http://migdal.wikidot.com/en:collective-decoherence}{thesis}, grade: 5/5 \hfill { Jan 2011}
      \item Bachelor's Degree, Mathematics (Game Theory),
      \href{http://migdal.wikidot.com/en:mafia}{thesis}, grade: 5/5 \hfill { Sept 2009}
    \end{list2}



%\vspace{-2mm}


    %____________________________________________________________________________________
    % Activities
    \section{\mysidestyle Teaching\\Gifted\\High-School\\Students}
    \begin{list2}
        \item Courses on: quantum mechanics and cryptography, simulating social phenomena in Python, measuring fractal dimension from images, Fourier analysis and waves
        \item Voluntary tutor of the \href{http://www.fundusz.org/?lang=gb}{Polish Children's Fund} \hfill { 2006--}
        \item Mentor of the winners of the \href{http://www.eucys09.fr/}{21st European Union Contest for Young Scientists}\hfill { 2009}
    \end{list2} 

%____________________________________________________________________________________
    % Activities
    \section{\mysidestyle Projects,\\Activities}

    \begin{list2}
    \item Started an unconference series \href{http://offtopicarium.wikidot.com/}{Offtopicarium} \hfill { Jan 2012--}
    \item Designing an educational quantum game, with ICFO and CosmoCaixa \hfill { Mar 2012--}
    \item Co-founder of \href{https://confrenzy.com}{Confrenzy} --- a website listing scientific events \hfill { Mar 2011--2012}

    \item Co-organized \href{http://warsztatywww.wikidot.com/en}{6th and 7th Summer Scientific Schools} \hfill { 2010--2011}
    \item The head organizer of a national physics student's conference VII OSKNF \hfill { 2008}
    \item Founded a student's association \href{http://skfiz.fuw.edu.pl/en}{SKFiz UW} \hfill { 2006}
    \end{list2}

%\vspace{-2mm}

   %____________________________________________________________________________________
    % Awards and Scholarships
    \section{\mysidestyle Awards,\\Scholarships}
    \begin{list2}
    	\item ICFO Innovation Fund grant awarded for Confrenzy \hfill { Oct 2011}
        \item TEAM Program operated by the Foundation for Polish Science \hfill { 2010}
        \item National Scholarship for exceptional achievements in science \hfill { 2007--2010}
  %      \item 2nd  place in the Didactic Show Competition (Faculty of Physics, University of Warsaw)  \hfill 2008  %6.12.2008 
        \item Research Science Institute (Massachusetts Institute of Technology, MA, USA) \hfill { 2005}
        \item Scholarship of the Polish Children's Fund \hfill { 2003--2005}
 %       \item Scholarship of the Minister of Education, Science and Sport \hfill 2003--2005
        \item Bronze Medal in the International Physics Olympiad (Pohang, South Korea) \hfill { 2004}
  %      \item 2nd  place in the 53rd Polish Physics Olympiad \hfill 2004
  %      \item 7th  place in the 52nd Polish Physics Olympiad \hfill 2003
    \end{list2}

%\vspace{-2mm}

    %____________________________________________________________________________________
    % Skills
    \section{\mysidestyle Skills}
    \begin{list2}
        \item Languages: Polish (native), English (fluent)
        \item Programming languages: Python, JavaScript, Mathematica, LaTeX,\\
        previously: MATLAB, LabView, Ocaml, PostScript
        \item OS (user): Mac OS X, Linux, Windows
    \end{list2}

%\vspace{3mm}
    %____________________________________________________________________________________
    % Hobbies
    %\section{\mysidestyle Hobbies}
    %photography, hiking, cognitive science

%________________________________________________________________________________________
\end{resume}
\end{document}

%________________________________________________________________________________________
% EOF
